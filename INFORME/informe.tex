\documentclass[11pt]{article}
\usepackage[spanish]{babel}
\usepackage{amsmath, amssymb, bm}
\usepackage[a4paper,margin=2.5cm]{geometry}

\newcommand{\ii}{\mathrm{i}}

\title{Tarea 1 Tecnologías Cuánticas}
\author{Lukas Wolff C.\\
        Patricio Palacios\\
        Juan Artigas\\
        Nicolas Mora\\
        Antonia Dias\\
        Benjamin Tapia}
\date{}

\begin{document}
\maketitle

% =========================
\section{Scripts de Bob y Alice}

Se han desarrollado dos scripts, \texttt{alice\_choices.py} y \texttt{bob\_choices.py}, que generan las elecciones de bases para Alice y Bob, respectivamente. Cada script produce ángulos aleatorios en el rango de $-22.5^\circ$ a $22.5^\circ$. Para asegurar la reproducibilidad de los resultados, se utilizan semillas distintas: 1001 para Alice y 2002 para Bob.

Los archivos funcionan de la siguiente manera:

\section{Script del Árbitro}

El árbitro ejecuta un script que procesa las mediciones de Alice y Bob, asignando valores de $+1$ o $-1$ según las siguientes reglas:

\begin{enumerate}
    \item \textbf{Bases iguales:} Si Alice y Bob miden en la misma base, ambos obtienen el mismo valor ($+1$ o $-1$).
    \item \textbf{Bases diferentes:} Si las bases son distintas, los valores de Alice y Bob se asignan de manera independiente ($+1$ o $-1$).
\end{enumerate}

A continuación, se suman los resultados de cada participante y se calcula la correlación entre ambos conjuntos de datos.

\textbf{Interpretación de la correlación:}
\begin{itemize}
    \item Si la correlación está en el rango $2 < \text{correlación} < 2.82$, se concluye que existe entrelazamiento cuántico.
    \item Si la correlación es menor o igual a $2$, se asume que el sistema está en un estado clásico.
\end{itemize}



\end{document}
