\documentclass[11pt]{article}
\usepackage[spanish]{babel}
\usepackage{amsmath, amssymb, bm}
\usepackage[a4paper,margin=2.5cm]{geometry}

\newcommand{\ii}{\mathrm{i}}

\title{Tarea 5 Tecnologías Cuánticas}
\author{Juan Artigas,
    Antonia Dias,
    Lukas Wolff,
    Patricio Palacios,
    Manuel Tagle,
    Benjamin Tapia}
\date{}

\begin{document}
\maketitle

% =========================
\section{Scripts de Bob y Alice}

Se han desarrollado dos scripts, \texttt{alice\_choices.py} y \texttt{bob\_choices.py}, que generan las elecciones de bases para Alice y Bob, respectivamente. Estos scripts seleccionan específicamente los ángulos necesarios para el test CHSH: $0^\circ$ y $45^\circ$ para Alice, y $22.5^\circ$ y $-22.5^\circ$ para Bob. Para asegurar la reproducibilidad de los resultados, se utilizan semillas distintas: 1001 para Alice y 2002 para Bob.

Los archivos funcionan de la siguiente manera:

\section{Script del Árbitro}

El árbitro ejecuta el script \texttt{referee\_quantum.py}, que implementa el test de Bell usando el criterio CHSH (Clauser-Horne-Shimony-Holt). El proceso se desarrolla de la siguiente manera:

\subsection{Generación de Resultados Cuánticos}

Para cada par de mediciones con ángulos $\theta_A$ (Alice) y $\theta_B$ (Bob), el script simula un estado entrelazado Bell $|\Phi^+\rangle = \frac{1}{\sqrt{2}}(|00\rangle + |11\rangle)$ mediante la función \texttt{sample\_joint()}, que:

\begin{enumerate}
    \item Calcula la correlación teórica: $E = \cos(2(\theta_A - \theta_B))$
    \item Define las probabilidades cuánticas para los cuatro resultados posibles:
    \begin{align}
        P(+1,+1) &= P(-1,-1) = \frac{1+E}{4} \\
        P(+1,-1) &= P(-1,+1) = \frac{1-E}{4}
    \end{align}
    \item Muestrea aleatoriamente un resultado según estas probabilidades
\end{enumerate}

\subsection{Cálculo de Correlaciones}

El script calcula cuatro correlaciones específicas $E(a,b)$ para las combinaciones de bases del test CHSH:

\begin{itemize}
    \item $E(a,b)$: bases $a = 0^\circ$ y $b = 22.5^\circ$
    \item $E(a,b')$: bases $a = 0^\circ$ y $b' = -22.5^\circ$
    \item $E(a',b)$: bases $a' = 45^\circ$ y $b = 22.5^\circ$
    \item $E(a',b')$: bases $a' = 45^\circ$ y $b' = -22.5^\circ$
\end{itemize}

Para cada combinación de bases, la correlación se calcula como:
$$E(a,b) = \langle A_k \cdot B_k \rangle = \frac{1}{N} \sum_{k=1}^{N} A_k \cdot B_k$$

donde $A_k$ y $B_k$ son los resultados ($\pm 1$) de Alice y Bob para el par $k$, y $N$ es el número de mediciones para esa combinación específica de bases.

\subsection{Parámetro CHSH}

El parámetro CHSH se calcula como:
$$S = E(a,b) + E(a,b') + E(a',b) - E(a',b')$$

\textbf{Interpretación del parámetro CHSH:}
\begin{itemize}
    \item $|S| \leq 2$: Compatible con teorías de variables ocultas locales (límite clásico)
    \item $2 < |S| \leq 2\sqrt{2} \approx 2.828$: Violación de desigualdades de Bell, evidencia de entrelazamiento cuántico
    \item $|S| > 2\sqrt{2}$: Imposible según la mecánica cuántica (límite de Tsirelson)
\end{itemize}

\section{Resutados}

\subsection{Bases Iniciales}

Sobre las bases iniciales de Alice y Bob, se obtiene una correlacion de 2.77 lo que indica que se cumple el limite inferior de 2 y el superior de 2.828, lo que indica que tales bases se encuentran entrelazadas.

A continuacion se realizara el analisis sobre un conjunto de bases, donde se estudiara como se ven afectadas las correlaciones y el parametro CHSH.

\section{Notas}

Es importante notar que para todas las bases se mantuvieron las bases tanto para Alice y Bob, asi como, el numero de mediciones (1000).

\end{document}